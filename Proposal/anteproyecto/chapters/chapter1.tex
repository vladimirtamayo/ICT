
\chapter{Definición del Problema y Justificación}

Pedro es un adolecente tímido a quien le gustan las matemáticas. Actualmente cursa el grado noveno, pero últimamente no participa mucho en clase. Todo esto comenzó a raíz de que, hace algunas semanas, se interesó por profundizar en un tema. Durante la clase de matemáticas, le pidió ayuda al profesor con un problema que había despertado su interés. Sus compañeros empezaron a hacerle comentarios ofensivos porque pensaron que les asignarían más tarea. A pesar de que el caso de Pedro es hipotético describe una realidad que afecta a una gran cantidad de personas durante sus procesos de aprendizaje, esto sumado a las difíciles condiciones del entorno educativo colombiano que empeoraron durante la pandemia de Covid-19 como la falta de acceso o acceso precario a Internet, poco conocimiento en el manejo de herramientas tecnológicas y la carencia de hábitos de aprendizaje autónomo \cite{VisDificultadesMonteroMahecha_2021}.

La ONU afirma \cite{ONU2024} en los objetivos de desarrollo sostenible que "se estima que en 2030 unos 300 millones de niños y jóvenes seguirán careciendo de conocimientos básicos de aritmética y alfabetización", lo cual pone de manifiesto que las problemáticas asociadas con la educación no son específicas del entorno colombiano. Esto demuestra la necesidad de impulsar la creación de herramientas que permitan el acceso a educación de calidad a bajo costo. Desde el año 2022, se ha observado un gran crecimiento tanto en el desarrollo como en el uso de grandes modelos de lenguaje (LLM) para realizar tareas para las cuales no fueron inicialmente concebidos, las llamadas "habilidades emergentes". Esto permite pensar que los grandes modelos de lenguaje pueden tener aplicaciones en educación. El presente proyecto busca determinar cómo la inteligencia artificial generativa puede contribuir al desarrollo de sistemas de tutores inteligentes de una forma económica y aplicable en una amplia variedad de entornos, sin pretender reemplazar a los tutores humanos. Además, cabe destacar que la efectividad de los tutores personalizados en procesos de aprendizaje ha sido evidenciada desde hace varias décadas \cite{BLOOM1984}, pero debido a los altos costos de un tutor humano, se hace muy difícil que la gran mayoría de los estudiantes puedan acceder a este tipo de tutorías.
